% In preamble
\renewcommand{\O}{\mathcal{O}}
\newcommand{\W}{\mathcal{W}}
\newcommand{\Piit}{\mathit{\Pi}}

\usepackage{physics}

S(t) := \text{Underlying asset price} \\
T := \text{Exercise date} \\
\mathcal{C}(S, T) := \text{Payoff of a call option for asset S at time T} \\
\mathcal{P}(S, T) := \text{Payoff of a put option for asset S at time T} \\
\mathcal{O}(t) := \text{Option price} \\
P(S, t) := \text{Underlying asset price} \\

% Eqn 5.5
\dd{S} = \mu S(t) \dd{t} + \sigma S(t) \dd{W(t)}

% Eqn 5.15
\begin{eqnarray*}
\W(t)               & = & \text{Wealth possessed by the writer at time $t$.} \\
\Delta(t)           & = & \text{Fraction of the underlying kept as safety.}\\
\mathit{\Pi(t)}     & = & \text{Cash amount kept as safety at time t.}\\
\W (t)              & = & \Delta (t)S(t) + \mathit{\Pi (t)} = \O(t)
\end{eqnarray*}

% Eqn 5.14
Itô process:
\begin{eqnarray*}
\dd{\O} & = & \left( \pdv{\O}{t} + \mu S(t) \pdv{\O}{S} + \frac{1}{2} \left( \sigma S(t) \right)^2 \pdv[2]{\O}{S} \right) \dd{t} + \sigma S(t) \pdv{O}{S}\dd{W(t)} \\
& = & \left( \pdv{O}{t} + \frac{1}{2}\left( \sigma S(t) \right)^2 \pdv[2]{\O}{S} \right) \dd{t} + \pdv{\O}{S}\dd{S}
\end{eqnarray*}

% Eqn 5.17
Delta hedging:
\begin{equation*}
    \dd{\O} = \Delta (t) \dd{S} + \dd{\Piit} = \Delta(t)\dd{S} + r\Piit \dd{t}
\end{equation*}

% Eqn 5.18
Therefore:
\begin{eqnarray*}
    \Delta(t) &=& \pdv{\O}{S} \\
    r\Piit    &=& \pdv{\O}{t} + \frac{1}{2} \left( \sigma S(t) \right)^2 \pdv[2]{\O}{S}
\end{eqnarray*}

Finally:
\begin{equation*}
    \underbrace{\pdv{\O}{t} + \frac{1}{2} \left( \sigma S(t) \right)^2 \pdv[2]{\O}{S}}_{r \Piit} + \underbrace{r \pdv{\O}{S} S}_{r \Delta(t) S} - r \O = 0
\end{equation*}

The portfolio has the value:
\begin{equation*}
    \mathit{\Pi}(t) &=& S(t) + \mathcal{P}(S,t) - \mathcal{C}(S,t) \\
\end{equation*}

At expiry, the options yield:
\begin{eqnarray*}
    \mathcal{C}(S,T) &=& \text{max}(S(T) - K,0) \\
    \mathcal{P}(S,T) &=& \text{max}(K - S(T),0) \\
\end{eqnarray*}

So the payoff of the portfolio is:
\begin{eqnarray*}
    \text{if} \: S \geq K: \quad &S& + 0 - (S - K) = K \\
    \text{if} \: S \leq K: \quad &S& + (K - S) - 0 = K \\
\end{eqnarray*}

Since the portfolio has a risk free profit, it's equivalent to the risk-free placement:
\begin{equation*}
    \mathit{\Pi(t)} = Ke^{-r(T-t)} \\
\end{equation*}

Otherwise there's arbitrage:
\begin{eqnarray*}
    \text{If $\mathit{\Pi}(t) = \epsilon Ke^{-r(T-t)}$ with $\epsilon \neq 1$}
    \text{if} \: \mathit{\Pi(t)} < Ke^{-r(T-t)}: \\
    K - \mathit{\Pi(t)}e^{r(T-t)} &=& K - (\epsilon Ke^{-r(T-t)})e^{r(T-t)} \\
                                  &=& K(1-\epsilon) \quad (\epsilon < 1)\\
    \text{if} \: \mathit{\Pi(t)} > Ke^{-r(T-t)}: \\
    \mathit{\Pi}(t) - Ke^{-r(T-t)} &=& (\epsilon - 1)Ke^{-r(T-t)} \quad (\epsilon > 1) \\
\end{eqnarray*}

\begin{equation*}
    \mathcal{P}(S,t) = \mathcal{C}(S,t) - S(t) + Ke^{-r(T-t)}
\end{equation*}
